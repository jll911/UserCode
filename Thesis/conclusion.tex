\chapter{Summary and Conclusions}
\label{summ}

This analysis measured the cross section of the 
$p\bar{p} \rightarrow Z \rightarrow e^+ e^-$ 
interaction within an invariant mass window of 60 to 120 GeV, 
using 36.1 \pb of $\sqrt{s} = 7$-TeV $pp$ collision data.  
Each event was required to have passed the trigger 
and to have two well-reconstructed electrons, 
which resulted in 8453 events.  
The event selection efficiency was determined using the 
tag-and-probe method to be 0.610, 
and the acceptance was calculated using a 
POWHEG (NLO) Monte Carlo \Zee sample to be 0.387.  
Background was estimated from PYTHIA Monte Carlo samples, 
from a template method using signal- and background-like 
samples, 
and from a functional fit to the 
candidate invariant mass distribution.  
Systematic uncertainties were estimated 
by varying the electron energy scale, 
varying the Monte Carlo sample used for the efficiency ratio, 
and taking errors on the fit where fits were performed.  
The error on the acceptance due to theoretical uncertainties 
was taken from the official CMS analysis.  

The cross section was determined to be 
\[
\sigma(p \bar{p} \rightarrow Z/\gamma *) \times \mathrm{BR}(Z \rightarrow e^+ e^- )
= 990 \pm 40 \mathrm{(lumi)} \pm 11 \mathrm{(stat)} \pm 17 \mathrm{(theory)} \pm 17 \mathrm{(syst)} \mathrm{pb}
\]
This result is in agreement with the FEWZ (NNLO) 
prediction of 972 $\pm$ 40 pb.  
In addition, it agrees with the CMS official measurement of 992 pb, 
as well as with the ATLAS measurement of 972 pb.  

The agreement of these results bodes well for CMS: 
the detector is well calibrated and can produce 
a solid measurement of standard model physics.  
This points toward a future of successful new 
measurements and exciting physics.  

% And I have learned a HECK of a lot.  Mission accomplished, I guess. 