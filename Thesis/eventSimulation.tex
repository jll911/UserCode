\chapter{Event Simulation}
\label{sim}
%somewhere have the pretty picture from 
%matt mc tutorial on how generators/geant/etc 
%fit in with everything else (reco, whatever)

%ARXIV PAPER OF AWESOMENESS
%http://arxiv.org/abs/hep-ph/0403045

%PYTHIA MANUAL OF AWESOMENESS
% the one I'm looking at is lutp0613man2.pdf, 
% but there's probably a more recent one




% I THINK I HAVE TOO MANY LONG SENTENCES 
% AND PREPOSITIONAL PHRASES


%need for event simulation: 

Within %the role of % CHANGE THIS CONSTRUCTION
a high-energy physics experiment's role of making 
discoveries and measurements, 
it necessary to know ahead of time what exactly 
should be expected. 
What will the detector signature of a new process look like? 
How will we know whether we've seen something unexpected? 
This is where event simulation contributes.  
%what exactly is the simulated data
Essentially, a series of programs is used to carefully 
generate and calculate all the relevant quantities for a set of 
fake events.  
This information is then used 
to aid analysis in ways for which 
real data alone cannot suffice.  

Having the ability to simulate physics processes 
serves multiple purposes. 
It aids in detector design: 
knowing the expected typical characteristics of particle 
interactions is essential to design a detector 
suited for those interactions.  
Once the detector design is settled, 
simulation of the detector is useful to design 
the algorithms used to reconstruct particles 
from their signature interactions with the 
detector material.  
In addition, simulating the physics processes 
can give estimates of how many 
events of a particular type are expected, 
further aiding the design process.  
The event simulation also contributes 
directly to many analyses, 
in the way of calculating the acceptance, 
the fraction of events that can theoretically 
be detected (see Section~\ref{FIXME}): 
it is impossible to know from observation 
how many events are missed by the detector, 
because of the very fact that they are unseen.  
It is instead necessary to get that fraction 
from a framework in which 
the characteristics of all events are 
inherently known, not reconstructed.  
Finally, % and possibly most visibly (?) I don't like that
simulated data is directly compared 
with ``real'' data from the detector 
to interpret the real-data results.  
If the data shows something significantly 
different from the simulation, 
then something is missing: 
perhaps a calibration needs to be applied, 
or the response of a detector unit needs 
to be further understood.  
Or, perhaps, there is a new physics process 
appearing for the first time, 
which was not previously known and 
which was therefore not present in the simulation.  
In this case simulation of proposed new physics 
processes may narrow down the identity of the 
observed new process.  
Whatever the case, 
discrepancy between observed data and the event 
simulation indicates that further 
investigation is necessary.  


%points of event simulation: 

%   * directly relevant: compare with data (description of current understanding as is) 
%-- we want to be able to interpret what we're seeing 
%in terms of what physics process it might be

%   * also relevant: acceptance is necessarily a MC calculation!

%   * also: design detectors, design reconstruction and other (e.g. trigger) algorithms

%   * also: event rates (like trigger rates?)  

The entire detection process is simulated, 
including the protons' direct interaction 
and any subsequent particle decays, 
as well as how the end-product particles 
interact with the detector as they pass through 
and how the detector itself responds.  
This latter part includes not only the 
material of the individual subdetectors 
but also the algorithms of 
the Level-1 Trigger, 
which are implemented in hardware 
(see Section~\ref{exp:L1}).  
The High-Level Trigger does not need to be 
simulated in this way; 
since its algorithms are all software-based, 
the same code can be run without modification 
on both real data and simulated data.  



\section{Monte Carlo Event Generation}
\label{sim:MC}
put why called ``monte carlo''

\subsection{Put this stuff in its own section?}
\label{sim:MCexplain}

types of programs with different types of output -- what they're used for. 
start with tree-level stuff and dsigma eqn...

   * cross section integrators

   * event generators

tree-level all well and good, but need corrections to really 
make things realistic.  
ways of doing the corrections:

   * matrix element

   * parton shower

And how these two things fulfill the different needs for 
both higher orders and hadronization

HOW THE PROGRAMS USED TO GENERATE THESE SAMPLES AND NUMBERS 
FIT INTO THIS FRAMEWORK.  
   pythia, powheg, tauola, fewz... else??  (didn't end up using MCatNLO) 
I don't think anything else.  
also explain about fewz different (xsec integrator, sounds like)

AND, FEWZ webpage was on frank petriello's wisconsin space and doesn't exist anymore

http://www.phys.hawaii.edu/~kirill/FEHiP.htm links to 
http://www.hep.wisc.edu/~frankjp/FEWZ.html (bad link)

FEWZ 2.0 abstract on the arXiv: http://arxiv.org/abs/1011.3540 
from ryan gavin, ye li, frank petriello, and seth quackenbush (15 Nov 2010)

WHICH VERSION DID VBTF USE??  look at syst AN.  
e-mail frank about documentation page??

https://indico.cern.ch/getFile.py/access?contribId=38\&resId=3\&materialId=slides\&confId=71330  tutorial from ryan giving original arXiv references (non-2.0)

pythia is definitely parton shower, 
explain steps and hadronization model, etc

%\subsection{some explanation of the black box?}
%\label{sim:MCBlackBox}
% it's not really a black box anymore!
%can talk about black-box mentality in introduction
modular structure here?  that pretty picture showing all the 
different steps to an interaction that need to be 
simulated. 
all those steps can be done by separate packages, 
which is the philosophy.  (pythia manual)
pdfs ->
hard scatter (underlying event, multiple interactions in parallel) ->
[parton] shower (both isr and fsr, incl QED) ->
hadronization (not directly applicable to my process, but applicable for bg's) -> 
decay resonances -> interactions with detector material ...

\subsection{Monte Carlo Generator Programs}
\label{sim:MCGens}

\subsubsection{PYTHIA}
\label{sim:MCGensPythia}
what are the different tunes for?  tune z2, tune d6t

\subsubsection{POWHEG}
\label{sim:MCGensPowheg}
why is this one used in favor of pythia?  

\subsubsection{Other Generators}
\label{sim:MCGensOther}
Like, whichever other samples I check (background etc)



\section{Detector Simulation}
\label{sim:Detector}
%\subsection{GEANT Detector Model/Modeling of Particle Interactions in Detector Material}

\subsection{GEANT Detector Model}
\label{sim:DetectorGeant}

\subsection{Level-1 Trigger Emulator}
\label{sim:DetectorL1Emul}
talk about how emulator necessary for l1, but not for hlt (duh)
