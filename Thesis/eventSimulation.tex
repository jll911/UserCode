\chapter{Event Simulation}
\label{sim}
somewhere have the pretty picture from 
matt mc tutorial on how generators/geant/etc 
fit in with everything else (reco, whatever)

points of event simulation: 
compare with data (desciption of current understanding as is) 
design detectors

\section{Monte Carlo Event Generation}
\label{sim:MC}
put why called ``monte carlo''
multiple types: shower vs matrix element?
\subsection{some explanation of the black box}
\label{sim:MCBlackBox}
modular structure here?  that pretty picture showing all the 
different steps to an interaction that need to be 
simulated. 
all those steps can be done by separate packages, 
which is the philosophy.  (pythia manual)
pdfs ->
hard scatter (underlying event, multiple interactions in parallel) ->
[parton] shower (both isr and fsr, incl QED) ->
hadronization (not directly applicable to my process, but applicable for bg's) -> 
decay resonances -> interactions with detector material ...
\subsection{Monte Carlo Generators}
\label{sim:MCGens}
\subsubsection{PYTHIA}
\label{sim:MCGensPythia}
what are the different tunes for?  tune z2, tune d6t
\subsubsection{POWHEG}
\label{sim:MCGensPowheg}
why is this one used in favor of pythia?  
\subsubsection{Other Generators}
\label{sim:MCGensOther}
Like, whichever other samples I check (background etc)

\section{Detector Simulation}
\label{sim:Detector}
%\subsection{GEANT Detector Model/Modeling of Particle Interactions in Detector Material}
\subsection{GEANT Detector Model}
\label{sim:DetectorGeant}

\subsection{Level-1 Trigger Emulator}
\label{sim:DetectorL1Emul}
talk about how emulator necessary for l1, but not for hlt (duh)
