\chapter{Overview}
\label{over}
\section{Measurement of Cross Section}
\label{over:xsec}

   * particles?  although that's theory intro (but need SOME particle-specific stuff) 

%the practical purpose of MC -- can simulate all stages of gen/sim/reco/etc and compare 
%everything step-by-step with data, as well as of course getting some quantities 
%that require extra knowledge, like acceptance (and predicting efficiencies, rates, etc)
%HA, that sounds like a few of those things in the pythia manual intro!  and I came 
%up with them myself!  

%glossary? (explanatory text, with true glossary in appendix?)  

Introductory terminology?  put terminology section in beginning?

want to talk about Z and why important, motivations

   * events
In particle physics, an ``event'' is a particle interaction 
that has been captured and recorded by the detector.  
%   * signal, background.  
In general, %events that contribute to the specific process     
events caused by the specific process     
being studied are termed ``signal,'' 
while any events 
%that do not contribute 
from other sources 
are called ``background.'' 
A significant part of any analysis is designing 
criteria that select signal events while 
rejecting background events, 
so that the analysis focuses only on data containing 
that process.  
The number of events for a given signal can be used 
to calculate quantities of interest, 
for example how often that process  
occurs relative to others.  

   * cross section!
The cross section, the quantity being calculated in this analysis, 
is such a measure of ``how often things happen.''  
However, it's not measured as a rate (occurrences per unit time), 
it's measured as a cross-sectional area 
and represents the probability of the given interaction occurring.  
An analogy may be made to trying to hit a target with a tennis ball.  
The bigger the target is, the more likely you will be able to hit it.  
The sizes of both objects matter: 
a trajectory that would cause a tennis ball to just miss the target 
would cause a basketball to clip the edge, 
solely because of the basketball's larger size.  
Therefore ``cross section'' may be more precisely defined as the 
effective cross-sectional area of the target, 
taking into account the sizes of both the target and the projectile.  
When both the target and projectile are particles, %such as protons, 
they may interact ``at a distance,'' that is, without ``touching'' each other, 
through the fundamental forces.  
(For example, electrons are thought to be point particles 
and therefore have no spatial extent, 
but they still attract and repel other particles 
through the electromagnetic force.) 
This interaction-at-a-distance increases the effective cross section.  
An interaction involving two particles that 
interact with each other very strongly has a large cross section.  
The cross section depends on the strength of the force between them.  
FORMULA AND EXPLANATION HERE??

According to the definition, ``interaction cross section'' applies only to 
the particles doing the colliding.  
However, a cross section is usually associated with the entire process, 
such as (in our case) 
$ q\bar{q} \rightarrow Z/ \gamma^{*} \rightarrow e^{+} e^{-} $.  
Here, the number calculated as the cross section takes into account 
only those events where a \Zg is formed, 
as well as the fact that only some of those events decay into electrons.  
The fraction of events that decay to a certain final state is known as 
the branching ratio, abbreviated BR.  
When it is not explicitly mentioned along with the cross section 
for a given final state, it is understood to be included.  
Adding up the cross sections for all of these possible interactions 
would give the full proton-proton interaction cross section.  

%xsec(Z) = n(Zee) / ( L * BR(Zee) * eff )
\[
\sigma_{ Z } \times \textrm{ BR }_{ Zee } = \frac{ n_{ Zee } }{  \mathcal{ L } \times \epsilon \times A}
\]
where
\[
%\epsilon = \epsilon_{ acc } \times \epsilon_{ trig } \times \epsilon_{ reco } \times \epsilon_{ sel }
\epsilon = \epsilon_{ trig } \times \epsilon_{ reco } \times \epsilon_{ sel }
\]

   * acceptance, efficiency (especially for that formula above)


   * ``data-driven'' stuff?  need for efficiency, but also systs??
The term ``data-driven'' is used to describe an analysis method 
that uses only real data, taken with the detector, 
instead of relying on simulated data.  
The purpose of using a data-driven method is to 
eliminate the possibility of a physics result 
being affected by an error or inaccuracy 
in the simulation itself.  
Since some quantities, such as efficiency, 
are much easier to measure using simulated data, 
devising data-driven methods to measure those quantities 
is an important and potentially challenging 
part of analysis.  

\section{Anatomy of Event}

%Like, what actually happens!

%bunches of protons collide, maybe two protons interact very strongly
%define parton

%A detector for a particle accelerator 
%records 
A proton-proton interaction begins 
with beams of protons accelerated 
in opposite directions.  
The protons are formed into ``bunches,'' 
with many, many millions of protons per bunch.  
The beams of proton bunches are directed to 
cross at the center of the detector.  
%Most such ``bunch crossings'' do not result 
%in a significant interaction.  % interaction at all? YES, DUH pileup
Each such ``bunch crossing'' results in 
a very small number of proton interactions, 
the vast majority of which do not interact 
strongly enough to be interesting.  
However, sometimes a parton, 
i.e. a quark or gluon inside the proton, 
interacts very strongly with a parton 
from another proton.  
These are the types of interactions 
that are typically interesting.  
%   * ``hard'' vs ``soft''
%A ``hard'' interaction is one in which the partons 
%interact very energetically.  
An interaction in which the partons interact 
very energetically is a ``hard interaction.''  
Conceptually, the partons ``hit each other hard.''  
Hard interactions are typically detected 
by having end-product particles with 
a lot of momentum 
in the transverse direction, 
perpendicular to the protons' original 
direction of motion.  
Only in hard interactions is the original 
momentum disturbed so much; 
in ``soft,'' or low-energy, interactions, 
most of the protons' momentum 
continues in the same direction, 
down the beam-pipe.  

%exchange a particle
%produce another particle which then decays
Oftentimes the two partons exchange 
a particle, such as a photon or gluon.  
In other cases another particle is formed, 
such as a Z boson; 
this is the scenario 
studied in this analysis.  
Particles formed in this way 
are typically heavy and therefore 
short-lived, 
decaying into lighter, more stable particles.  
This analysis studies the Z's decay into two 
electrons, 
which are the lightest charged particles 
and therefore (to conserve charge) 
do not decay.  
These decay products are what fly ``out'' 
into the detector with 
some energy and direction, 
and it is these quantities 
that are measured by the detector.  

However, additional processes may contribute 
to the signature left in the detector.  
%
%ISR, FSR
One of the initial partons 
or final decay products 
may radiate an additional particle 
which then ends up in the detector.  
This is known as ``initial-state radiation'' (ISR) 
and ``final-state radiation'' (FSR) respectively.  
In addition, there are two ways that 
particles unrelated to the hard interaction 
may show up in the detector: 
underlying event and pileup.  
%
%   * underlying event!
The ``underlying event'' refers to the 
interactions taking place between the 
proton remnants, 
the parts of the proton ``left over'' 
from the hard interaction.  
%
%   * pileup.  maybe under, like, ``the anatomy of an event'' or ``the structure of an event?''
``Pileup'' refers to any interactions that happen 
between other protons in the bunch.  
These are generally soft, 
low-energy interactions.  
Statistically, at most only one hard interaction 
happens in any given bunch crossing.  
Both underlying event and pileup can contribute 
energy deposits that are recorded as part of the event.  
Typically they are both fairly low-energy and 
therefore easily filtered out as background.  

\section{Terms in particle physics, technical stuff}

%   * ``event''!  

   * ``high-energy''
``High-energy'' physics refers to modern particle physics. 
In the beginnings of particle physics, 
the available accelerators operated at much lower energies.  
Low-energy accelerators are still used in nuclear physics 
and aid in materials physics and biophysics studies. 
However, to probe the fundamental interactions between particles, 
today's accelerators use much higher energies.  

   * ``kinematic'' maybe previous section?
In the following pages, much will be described by the word 
``kinematic'' or ``kinematics''.  
This refers to the properties of position or motion.  
So, ``kinematic criteria'' means selections applied according to 
quantities such as momentum or angle.  


   * histogram?  errors? (stat/syst) etc?

   * statistical and systematic uncertainties or errors

   * ``phase space''! % -- move this to only place (so far) 
%that talks about it?  but theory WILL talk about it
``Phase space'' refers to the collection of total 
possible values for a set of quantities.  
A given system is represented by one set of 
values out of that collection.  
%Direction (in three dimensions)   % NOT ACTUALLY TRUE!!!
%plus speed or energy define a 
%four-dimensional phase space 
%for a particle's kinematic quantities.  
For some calculations it is necessary to include 
contributions from the entire phase space, 
for example integrating over all possible 
angles for a particle's direction.  

\section{coordinates used?}

   * center of momentum reference frame

For some high-energy physics experiments, 
there is a choice of several reference 
frames that can be used for measurement, 
each with their own benefits.  
%
However, in the case of proton-proton collisions, 
the logical reference frame for measurements is 
the frame at rest with respect to the detector, 
the ``lab frame,'' 
because it is the same as the 
proton-proton center of momentum frame.  
Therefore all measurements are done 
in the lab frame.  

%   * eta, phi, transverse quantities -- relate eta to y and all that, 
%talk about difference, talk about meaning of each one

A particular set of coordinates is often used to describe 
the direction of outgoing or intermediate particles.  
The direction of the incoming particles defines an axis, 
the beam axis.  
The azimuthal angle, measured around this axis 
and analogous to longitude on the earth's surface, 
is labeled $\phi$.  
The angle along the axis, the polar angle 
(analogous to latitude), 
is labeled by $\theta$.  
However, it is much more common to measure this 
angle in terms of pseudorapidity, $\eta$, 
which is given as a function of $\theta$:
\[
\eta = -\ln \left[ \tan \left( \frac{\theta}{2}\right) \right]
\]
$\eta$ is preferred because of its relation to another quantity, 
rapidity, $y$, which is a function of the particle's 
energy and momentum:
\[
y = \frac{1}{2} \ln \left( \frac{E+p_L}{E-p_L} \right)
\]
where $p_L$ is the longitudinal component of the 
particle's momentum (the component parallel to the beam axis).  
% WON'T DO COMPOSITION OF VECTORS
Rapidity is useful when dealing with relativistic speeds, 
such as those at which the particles typically travel.  
When viewed from separate reference frames that are moving 
at relativistic speeds with respect to each other, 
a particle's speed, direction, and energy appear to be different.  
However, rapidity has special status in that the 
difference in rapidity 
between two particles does not change 
when the reference frame is changed.  
A particle's rapidity can be approximated by the 
pseudorapidity if it is very light or 
traveling very fast 
(pseudorapidity is identical to rapidity 
if the particle has zero mass, 
in which case it is also 
moving at the speed of light).  
In addition, outgoing particles tend to be 
distributed uniformly in terms of rapidity, 
as opposed to the polar angle $\theta$, 
which does not enjoy the same uniform distribution. 
Since rapidity itself is not a measure of angle, 
pseudorapidity is used as the coordinate.  
Rapidity is calculated in cases 
where the particle of interest may be heavy 
and may not have a high speed, 
for example the Z boson examined in this analysis.  

\section{General Analysis programs}

%   * ROOOOOOOOOT!  other software, like CMSSW? yes

% root.cern.ch duh

% have no idea how to reference CMSSW -- workbook??

Two software programs in particular were 
used in this analysis.  

The CMS Software (CMSSW) package is 
the collaboration-developed software 
used for many aspects of CMS operation and use.  
It encompasses data-taking, reconstruction, 
simulation, and analysis.  
% what else to say?  

ROOT is a general data analysis package, 
written in C++, 
and developed and maintained at CERN. 
It is widely used in the field of 
experimental particle physics.  
ROOT implements many tools and functions 
necessary to various types of analysis, 
such as histograms, fitting, 
and statistics.  
CMSSW can interface with ROOT, 
providing access to ROOT's %wide 
%range of 
capabilities within a CMSSW 
analysis setting.  

