\chapter{Overview}
\label{over}
\section{Measurement of Cross Section}
\label{over:xsec}
%xsec(Z) = n(Zee) / ( L * BR(Zee) * eff )
\[
\sigma_{ Z } \times \textrm{ BR }_{ Zee } = \frac{ n_{ Zee } }{  \mathcal{ L } \times \epsilon \times A}
\]
where
\[
%\epsilon = \epsilon_{ acc } \times \epsilon_{ trig } \times \epsilon_{ reco } \times \epsilon_{ sel }
\epsilon = \epsilon_{ trig } \times \epsilon_{ reco } \times \epsilon_{ sel }
\]

particles?  although that's theory intro (but need SOME particle-specific stuff) 

the practical purpose of MC -- can simulate all stages of gen/sim/reco/etc and compare 
everything step-by-step with data, as well as of course getting some quantities 
that require extra knowledge, like acceptance (and predicting efficiencies, rates, etc)
HA, that sounds like a few of those things in the pythia manual intro!  and I came 
up with them myself!  

glossary? (explanatory text, with true glossary in appendix?)  

   * ``event''!  ``high-energy''

   * cross section!

   * histogram?  errors? (stat/syst) etc?

   * eta, phi, transverse quantities

   * ``kinematic''

   * I think I (hopefully) got minbias already, and this isn't exactly the place for it... 

   * signal, background

   * acceptance, efficiency (especially for that formula above)

   * invariant mass

   * HERE OR IN THEORY INTRO need to do on-shell/off-shell decays 
to explain why Z mass has a spectrum and not a signal value

   * statistical and systematic uncertainties or errors

   * ``data-driven'' stuff?  need for efficiency, but also systs??

   * acronyms?  

   * ROOOOOOOOOT!




really need these??  or is explaining them in text good enough?  (
I would think it's good enough.)

acronyms

   * CMS

   * LHC

   * ECAL/HCAL



glossary

   * tag and probe

   * template method?