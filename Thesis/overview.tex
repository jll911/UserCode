\chapter{Overview}
\label{over}
\section{Measurement of Cross Section}
\label{over:xsec}
%xsec(Z) = n(Zee) / ( L * BR(Zee) * eff )
\[
\sigma_{ Z } \times \textrm{ BR }_{ Zee } = \frac{ n_{ Zee } }{  \mathcal{ L } \times \epsilon \times A}
\]
where
\[
%\epsilon = \epsilon_{ acc } \times \epsilon_{ trig } \times \epsilon_{ reco } \times \epsilon_{ sel }
\epsilon = \epsilon_{ trig } \times \epsilon_{ reco } \times \epsilon_{ sel }
\]

particles?  although that's theory intro (but need SOME particle-specific stuff) 

the practical purpose of MC -- can simulate all stages of gen/sim/reco/etc and compare 
everything step-by-step with data, as well as of course getting some quantities 
that require extra knowledge, like acceptance (and predicting efficiencies, rates, etc)
HA, that sounds like a few of those things in the pythia manual intro!  and I came 
up with them myself!  

glossary? (explanatory text, with true glossary in appendix?)  

   * ``event''!  ``high-energy''

   * ``phase space''!

   * cross section!

   * here or in theory chapter define ``tree-level''.  
ALSO IN THEORY: explain how Feynman diagrams are so useful! 
writing down the matrix element and all that
AND PDFs, basically whole process of stuff simulated 
really happens and should be explained.  
AND DEF OF PARTONS
yeah, basically explain all the stuff in the MC chapter
INCLUDING ISR and FSR
ALSO define ``jets'' and talk about how they come from 
quarks and gluons (define partons) -- 
not entirely relevant for this analysis, 
but possible in general
AND define ``color''

   * histogram?  errors? (stat/syst) etc?

   * eta, phi, transverse quantities

   * pileup.  maybe under, like, ``the anatomy of an event'' or ``the structure of an event?''

   * ``kinematic''

   * ``hard'' vs ``soft''

   * I think I (hopefully) got minbias already, and this isn't exactly the place for it... 

   * signal, background

   * acceptance, efficiency (especially for that formula above)

   * invariant mass

   * HERE OR IN THEORY INTRO need to do on-shell/off-shell decays 
to explain why Z mass has a spectrum and not a single value

   * statistical and systematic uncertainties or errors

   * ``data-driven'' stuff?  need for efficiency, but also systs??

   * acronyms?  

   * ROOOOOOOOOT!  other software, like CMSSW?




really need these??  or is explaining them in text good enough?  (
I would think it's good enough.)

acronyms

   * CMS

   * LHC

   * ECAL/HCAL



glossary

   * tag and probe

   * template method?