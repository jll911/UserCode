\chapter{Results}
\label{res}

\section{Cross Section Measurement}
\label{res:xsec}
%Cross section calculation: the table of numbers, 
%and the formula

As explained in Section~\ref{over:xsec}, 
the interaction cross section is calculated from the formula 
\[
\sigma \times \mathcal{BR} 
= \frac{\left( n_{total} - n_{background }\right)}{\mathcal{ L } \times \epsilon \times A }
\]
in which $n_{total}$, $n_{background}$, $\mathcal{ L }$, $\epsilon$, and $A$  
have been measured experimentally.  
The measured values for these quantities 
are given in Table~\ref{TableXsecNumbers}. 

\begin{table}[htbp]
%  \centering
  \begin{center}
    \caption{Quantities used to calculate cross section.}
    \label{TableXsecNumbers}
%    \begin{tabular}[]{ | l | c | c | }
    \begin{tabular}[]{ | l | c | }
      \hline
      Quantity & Value \\ \hline \hline
      Luminosity $\mathcal{L}$ & 36.1 pb$^{-1}$ \\ \hline
      $n_{total}$ & 8453 \\ \hline
      $n_{background}$ & 19 \\ \hline
      Efficiency $\epsilon$ & 0.610 \\ \hline
      Acceptance $A$ & 0.387 \\ \hline
    \end{tabular}
  \end{center}
\end{table}



Plugging the values into the formula gives 
990 pb for the cross section.  
%
The sources of error in the measurement 
(see Section~\ref{anMeth:SystsSummary}) 
add a relative uncertainty of 4.8\%, 
corresponding to 48 pb, for a final result of 
\[
\sigma(p \bar{p} \rightarrow Z/\gamma *) \times \mathrm{BR}(Z \rightarrow e^+ e^- )
= 990 \pm 40 \mathrm{(lumi)} \pm 11 \mathrm{(stat)} \pm 17 \mathrm{(theory)} \pm 17 \mathrm{(syst)} \mathrm{pb} 
\]



\section{Comparison to Theory}
\label{res:theory}

To facilitate comparison of the cross section 
with theory, 
the CMS collaboration prepared a ``standard'' 
theory value with FEWZ 
(see Section~\ref{sim:MCGensOther}), 
using the appropriate detector boundaries 
for the acceptance.  
The corresponding value obtained was 
972 $\pm$ 40 pb.  
The experimental value, 990 $\pm$ 48 pb, 
agrees with the theoretical value within 
the errors on both values 
(the difference is 1.9\% relative 
to the theoretical value).  

%number from FEWZ, errors.  how experimental number compares within errors

\section{Comparison to Other Experiments}
\label{res:prev}

%official CMS number, 

The official CMS vector boson cross section analysis 
\cite{CMSWZ}  
calculated a value of 992 pb.  
The official CMS analysis was very similar 
to this one, 
so it is expected that the numbers should 
agree very closely.  
The difference between them comes to 0.2\%.  

The ATLAS also performed an analysis 
of the vector boson production cross sections 
\cite{ATLASZ}. 
The selection criteria were slightly different 
from those used in CMS: 
the $\eta$ range used was 
$|\eta| < 1.37, 1.52 < |\eta| < 2.47$, 
the electrons were required to have a 
transverse energy of 20 GeV instead of 25, 
and the invariant mass range was 66 to 116 GeV 
instead of 60 to 120 Gev.  
Nevertheless, the analyses are comparable.  
The \Zee cross section obtained by ATLAS is 
\[
\sigma(p \bar{p} \rightarrow Z/\gamma *) \times \mathrm{BR}(Z \rightarrow e^+ e^- )
= 972 \pm 33 \mathrm{(lumi)} \pm 10 \mathrm{(stat)} \pm 38 \mathrm{(theory)} \pm 34 \mathrm{(syst)} \mathrm{pb} 
\]
This compares well with the number obtained 
in this analysis, 
as well as with the official CMS result.  

%tevatron numbers??  doesn't really add anything

ADD FUN PLOTS, like move inv mass here?? 
also Z quantity plots. 
and do I have electron distributions somewhere? 
YES, they're in evSel.  along with Z distro plots.  
Those are really sort of results, I should put them there! 